%
% Annual Cognitive Science Conference
% Sample LaTeX Two-Page Summar -- Proceedings Format
%

% Original : Ashwin Ram (ashwin@cc.gatech.edu)       04/01/1994
% Modified : Johanna Moore (jmoore@cs.pitt.edu)      03/17/1995
% Modified : David Noelle (noelle@ucsd.edu)          03/15/1996
% Modified : Pat Langley (langley@cs.stanford.edu)   01/26/1997
% Latex2e corrections by Ramin Charles Nakisa        01/28/1997
% Modified : Tina Eliassi-Rad (eliassi@cs.wisc.edu)  01/31/1998
% Modified : Trisha Yannuzzi (trisha@ircs.upenn.edu) 12/28/1999 (in process)
% Modified : Mary Ellen Foster (M.E.Foster@ed.ac.uk) 12/11/2000
% Modified : Ken Forbus                              01/23/2004
% Modified : Eli M. Silk (esilk@pitt.edu)            05/24/2005
% Modified : Niels Taatgen (taatgen@cmu.edu)         10/24/2006
% Modified : David Noelle (dnoelle@ucmerced.edu)     11/19/2014
% Modified : Roger Levy (rplevy@mit.edu)     12/31/2018



%% Change "letterpaper" in the following line to "a4paper" if you must.

\documentclass[10pt,letterpaper]{article}

\usepackage{cogsci}

%\cogscifinalcopy % Uncomment this line for the final submission
\twopagesummarysubmission % switch for two-page summary submissions so
                          % that they're never anonymized


\usepackage{pslatex}
\usepackage{apacite}
\usepackage{float} % Roger Levy added this and changed figure/table
                   % placement to [H] for conformity to Word template,
                   % though floating tables and figures to top is
                   % still generally recommended!

\usepackage[none]{hyphenat} % Sometimes it can be useful to turn off
%hyphenation for purposes such as spell checking of the resulting
%PDF.  Uncomment this block to turn off hyphenation.


%\setlength\titlebox{4.5cm}
% You can expand the titlebox if you need extra space
% to show all the authors. Please do not make the titlebox
% smaller than 4.5cm (the original size).
%%If you do, we reserve the right to require you to change it back in
%%the camera-ready version, which could interfere with the timely
%%appearance of your paper in the Proceedings.



\title{How to Make a Proceedings Short Summary Submission}

\author{
  {\large \bf Alvin Wei-Ming Tan (@stanford.edu)} \\
  {\large \bf George Kachergis (kachergis@stanford.edu)} \\
  {\large \bf Michael C.~Frank (mcfrank@stanford.edu)} \\
  Department of Psychology, 420 Jane Stanford Way \\
  Stanford, CA 94305 USA}


\begin{document}

\maketitle


\begin{quote}
\small
\textbf{Keywords:}
theory development; psychometrics; measurement; individual differences
\end{quote}

\section{Introduction}


This workshop will explain...


\section{Goal and Scope}

This workshop will bring together cognitive scientists who have used psychometric models as a vehicle for understanding individual differences in diverse cognitive domains, from reasoning to language.
We have invited researchers who...

Main topics of discussion will be:
(bulleted list)


\section{Target Audience}

We expect that the topic of this workshop will be of broad appeal to the

\section{Organizers and Presenters}

{\em Alvin Wei-Ming Tan} is...
{\em George Kachergis} is a research scientist at Stanford University. He has studied language acquisition with psychometric models as well as process-based cognitive models of memory and learning, with a particular focus on self-directed, active learning.
{\em Michael C. Frank} is a Professor of Psychology at Stanford University. He...

\section{Workshop Structure}

We propose a full-day workshop consisting of four parts. Two parts will be a series of 20-minute talks and 5-minute discussions, outlined in Table 1.
After XXX's talk, we will lead groups of participants in thinking about how to apply psychometric models to their experimental paradigms of interest.
After the remaining talks, there will be a 45-minute panel discussion.


\begin{table}[H]
\begin{center}
\caption{Sample table title.}
\label{sample-table}
\vskip 0.12in
\begin{tabular}{ll}
\hline
Error type    &  Example \\
\hline
Take smaller        &   63 - 44 = 21 \\
Always borrow~~~~   &   96 - 42 = 34 \\
0 - N = N           &   70 - 47 = 37 \\
0 - N = 0           &   70 - 47 = 30 \\
\hline
\end{tabular}
\end{center}
\end{table}


\subsection{Figures}

All artwork must be very dark for purposes of reproduction and should
not be hand drawn. Number figures sequentially, placing the figure
number and caption, in 10~point, after the figure with one line space
above the caption and one line space below it, as in
Figure~\ref{sample-figure}. If necessary, leave extra white space at
the bottom of the page to avoid splitting the figure and figure
caption. You may float figures to the top or bottom of a column, and
you may set wide figures across both columns.

\begin{figure}[H]
\begin{center}
\fbox{CoGNiTiVe ScIeNcE}
\end{center}
\caption{This is a figure.}
\label{sample-figure}
\end{figure}


\section{Acknowledgments}

Place acknowledgments (including funding information) in a section at
the end of the paper.


\section{References Instructions}

Follow the APA Publication Manual for citation format, both within the
text and in the reference list, with the following exceptions: (a) do
not cite the page numbers of any book, including chapters in edited
volumes; (b) use the same format for unpublished references as for
published ones. Alphabetize references by the surnames of the authors,
with single author entries preceding multiple author entries. Order
references by the same authors by the year of publication, with the
earliest first.

Use a first level section heading, ``{\bf References}'', as shown
below. Use a hanging indent style, with the first line of the
reference flush against the left margin and subsequent lines indented
by 1/8~inch. Below are example references for a conference paper, book
chapter, journal article, dissertation, book, technical report, and
edited volume, respectively.

\nocite{ChalnickBillman1988a}
\nocite{Feigenbaum1963a}
\nocite{Hill1983a}
\nocite{OhlssonLangley1985a}
% \nocite{Lewis1978a}
\nocite{Matlock2001}
\nocite{NewellSimon1972a}
\nocite{ShragerLangley1990a}


\bibliographystyle{apacite}

\setlength{\bibleftmargin}{.125in}
\setlength{\bibindent}{-\bibleftmargin}

\bibliography{references}


\end{document}
